\chapter{Impostos sobre Investimentos: Otimizando Resultados}

\section{Imposto de Renda na Renda Fixa e Fundos}

\noindent Na maioria dos investimentos em renda fixa e fundos, o IR incide apenas sobre o rendimento (ganho de capital), seguindo a tabela regressiva:

\begin{center}
\begin{tabular}{|c|c|}
\hline
\textbf{Prazo de Aplicação} & \textbf{Alíquota de IR} \\
\hline
Até 180 dias & 22,5\% \\
\hline
De 181 a 360 dias & 20\% \\
\hline
De 361 a 720 dias & 17,5\% \\
\hline
Acima de 720 dias & 15\% \\
\hline
\end{tabular}
\end{center}

\vspace{0.3cm}
\noindent Exemplo: Se você aplicou R\$ 1.000 e resgatou R\$ 1.100 após 1 ano, seu rendimento foi de R\$ 100. A alíquota de IR será de 20\%, ou seja, R\$ 20 de imposto. Seu ganho líquido será de R\$ 80.

\section{IOF (Imposto sobre Operações Financeiras)}

\noindent O IOF incide sobre investimentos de renda fixa resgatados antes de 30 dias:

\begin{itemize}[leftmargin=*]
    \item A alíquota diminui a cada dia, partindo de 96\% (1º dia) até 0\% (30º dia).
    \item Incide apenas sobre o rendimento, não sobre o capital.
    \item Em fundos, o IOF só é cobrado para resgates feitos em menos de 30 dias.
\end{itemize}

\section{Resumo prático}
\begin{center}
\begin{tabular}{|p{4.8cm}|c|c|}
\hline
\textbf{Investimento} & \textbf{IR} & \textbf{IOF} \\
\hline
\textbf{LCI / LCA} & Isento & Isento (carência mínima \textbf{> 30 dias}) \\
\hline
\textbf{CRI / CRA} & Isento & Isento (carência mínima \textbf{> 30 dias}) \\
\hline
\textbf{Debênture Incentivada} & Isento & Isento (vencimento normalmente longo) \\
\hline
\textbf{Poupança} & Isento & Isento \\
\hline
\textbf{FII (rendimentos)} & Isento & Não se aplica \\
\hline
\textbf{Fundos de Ações} & 15\% & Isento \\
\hline
\end{tabular}
\end{center}